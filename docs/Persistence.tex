\title{The Agetac Build System}
\author{George Politis}

\documentclass[12pt]{article}

\begin{document}
\maketitle

\section{Introduction}

1. Why do we need an abstraction layer? Talk about how wonderful it is to write SQL or how portable your code is..
2. Present various ways to provide abstraction
3. Show which abstraction layer is the best (performance comparisons).

No abstraction at all (...), Hibernate, JDO, JPA. Provide performance comparisons between Hibernate and JDO. 

\section{JDO}

Java Data Objects (JDO) is a standard interface for storing objects containing data into a database. The standard defines interfaces for annotating Java objects, retrieving objects with queries, and interacting with a database using transactions. An application that uses the JDO interface can work with different kinds of databases without using any database-specific code, including relational databases, hierarchical databases, and object databases. As with other interface standards, JDO simplifies porting your application between different storage solutions.

JDO uses a post-compilation \emph{enhancement} step in the build process to associate data classes with the JDO implementation. TODO Reference the build system to show how we bytecode enhance our classes.

\section{Conclusions}

\end{document}
